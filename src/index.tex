% Gruppo per definizione di successione capitoli senza interruzione di pagina
\begingroup
  % Ridefinizione dei comandi di clear page
  % \renewcommand{\cleardoublepage}{}
  % \renewcommand{\clearpage}{}

  % Ridefinizione del formato del titolo del capitolo
  % Da
  %   Capitolo X
  %   Titolo capitolo
  % A
  %   X   Titolo capitolo
  \titleformat{\chapter}
    {\normalfont\Huge\bfseries}{\thechapter}{1em}{}
  % Spaziature dei titoli per varie sezioni
  \titlespacing*{\chapter}{0pt}{0.59in}{0.02in}
  \titlespacing*{\section}{0pt}{0.20in}{0.02in}
  \titlespacing*{\subsection}{0pt}{0.10in}{0.02in}

  % Sommario e' un breve riassunto del lavoro svolto dove si descrive
  % l’obiettivo, l’oggetto della tesi, le metodologie e
  % le tecniche usate, i dati elaborati e la spiegazione delle conclusioni
  % alle quali siete arrivati.
  % Il sommario dell’elaborato consiste al massimo di 3 pagine e deve contenere le seguenti informazioni:
  %  - contesto e motivazioni
  %  - breve riassunto del problema affrontato
  %  - tecniche utilizzate e/o sviluppate
  %  - risultati raggiunti, sottolineando il contributo personale del laureando/a
  \include*{sections/sommario}

  % Lista dei capitoli
  \include*{chapters/capitolo1}
  \include*{chapters/capitolo2}
  \include*{chapters/capitolo3}
  \include*{chapters/capitolo4}
  \include*{chapters/capitolo5}
  \include*{chapters/capitolo6}
\endgroup
