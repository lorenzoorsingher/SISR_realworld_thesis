\newpage
\chapter{Conclusions}
\label{cha:conclusions}

The final results of the project are very satisfactory and show how with just a small action camera and some little tricks it's possible to improve by a large margin and give new life to footage that would otherwise be wasted or deemed impractical for other applications. The entire process, from start to finish can be applied to basically any combination of two cameras, offering the possibility to create ad hoc image enhancement set-ups for a wide variety of situations, the final results may of course vary, but this proof of concept demonstrates how it's possible and practical.

The system is promising but it must be noted that there's a lot of room for improvement. First of all regarding data, unfortunately, data collection is a long and expensive process, both in terms of time and effort. Most of the videos recorded for this research were at the same location (albeit at different times of the day), consequently, the capabilities of the trained models are constrained to similar environments and conditions. Larger datasets could guarantee better generalization and make the models less prone to overfitting on the training data.

In conclusion, our objective was to set up a general pipeline that could allow anyone to create as effortlessly as possible a high-standard dataset with, otherwise very difficult to achieve, real-world distortions, and with that train a model able to work in concrete environments. We think all the above points have been achieved and the prospect of a possible expansion of the system is promising.
